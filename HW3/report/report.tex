\documentclass[a4paper,12pt]{article}   % 文檔類型為article
\usepackage{setspace}
\usepackage{fontspec}% 使用系统字体
\usepackage{xeCJK}    % 提供中文支持

% 设置中文正文字体为標楷體,你需要确保系统上有標楷體字体文件
\setCJKmainfont{標楷體}

%\onehalfspacing % 1.5 倍行距,與 Word 中的默認行距相同
%\pagestyle{empty}

\setmainfont{Times New Roman}
\fontsize{12pt}{\baselineskip}   % 設置字體大小為12pt,行距為單行


\usepackage{pdfpages}
\usepackage{enumitem}
\usepackage{fontspec}
\usepackage{authblk} 
\usepackage{titlesec}
\usepackage{amssymb}
\usepackage{graphicx}
\usepackage{titlesec}
\titlespacing{\section}{0pt}{*2}{*1}
\titlespacing{\subsection}{0pt}{*2}{*1}
\titlespacing{\subsubsection}{0pt}{*2}{*1}

\usepackage{tabularx}
\usepackage{multirow}
%\usepackage{colortbl}
\usepackage{amsthm}
\usepackage{booktabs} % 用於漂亮的表格樣式


\usepackage{colortbl} % 用於表格顏色
\usepackage{xcolor}   % 用於文本顏色
\definecolor{lightgray}{gray}{0.9} % 自定義顏色

\pagestyle{empty}


\begin{document}
%\maketitle

\begin{center}
	{\fontsize{16pt}{12pt}\selectfont Semi-Supervised Learning}
	
\end{center}

\hfill  B092040016 陳昱逢
	
	
\begin{center}
	Assignment 3
\end{center}

\section{模型架構設計}
	現今有許多卷積神經網路 (convolution neural network; CNN) 架構,他們一開始多用較大的卷積核 (kernel size) 先去取得初步特徵,但有論文提出通過堆疊多層 3*3 的卷積核也可以達到使用較大的卷積核獲得的大範圍接受域 (receptive field)。 因此本架構的發想都是基於多層 3*3 的卷積核堆疊而成,其中 conv4 接收的輸入除了來自 conv3 的 輸出之外,還會加上 conv2 的輸出,同理, conv5 輸入來自 conv4 的 輸出以及 conv3 的輸出,此想法來自於 ResNet\ \cite{he2015deep} 提出的 residual 概念。 Table\ \ref{table:arch} 呈現我發想的模型架構。
	
	
\begin{table}[htb]
	\centering	
	\normalsize
    \newcommand{\z}{\phantom{0}}
    \caption{Architectures of my CNN model}
    \vspace{0.1\baselineskip}
    \resizebox{0.6\textwidth}{!}{
		\begin{tabular}[c]{|l|l|}
			\hline
			\textbf{variable name} & \textbf{convolution architecture} \\
			\hline
			conv1   &  $3 \times 3$,\ 64 \\
    					& $3 \times 3$,\ 128 \\
    					& $3 \times 3$,\ 194 \\
			\hline
			Maxpool & $kernel\ size = 2$ \\
					& $stride = 2$ \\
			\hline
			conv2 & $3 \times 3$,\ 256 \\
			\hline
			conv3 & $3 \times 3$,\ 512 \\
			\hline
			conv4 & $3 \times 3$,\ 1024 \\
			\hline
			conv5 & $3 \times 3$,\ 2048 \\
			\hline
			Maxpool & $kernel\ size = 2$ \\
					& $stride = 2$ \\
			\hline
			FC      & 5-layer FC \\
    			\hline
		\end{tabular}
	}
	\label{table:arch}
   \vspace{0.1\baselineskip}
\end{table}



\section{實驗}

本實驗採用自訓練 (self-training) 的方式來完成半監督學習 (semi-supervised learning) 的任務,此部分將分為以下兩步驟去討論: 第 2.1 節 利用有標記的資料訓練和第 2.2 節 利用沒有標記的資料產生偽標記去訓練


\subsection{利用有標記的資料訓練}
	此步驟我資料擴增的方法採取了水平垂直翻轉、旋轉以及縮放等操作再進行訓練。在訓練時,我採取了 AdamW 作為我的 optimizer, 損失函數的評估方式為 Cross-entropy, 一開始我是使用 RMSProp 作為我的 optimizer, 但後來經過多次測試,從 RMSProp 換到 Adam 再換到目前所用的 AdamW, 收斂的速度都有依序變快,原因可能為 AdamW 考慮了 momentum 以及 L2 正則化, 可以提升模型的收斂速度以及泛化能力。此外,我也有採用了 ReduceLROnPlateau 的學習速率調整策略,他會根據我驗證集的準確率指標來調整學習速率,當我的驗證集的準確率指標停滯不前時就會降低學習速率。 Table \ref{table:config1} 呈現了我訓練時 optimizer 以及 learning rate scheduler 的參數
	
	
\begin{table}[htb]
	\centering	
	\normalsize
    \newcommand{\z}{\phantom{0}}
    \caption{Configurations of optimizer and learning rate scheduler}
    \vspace{0.15\baselineskip}
    \resizebox{0.7\textwidth}{!}{
		\begin{tabular}[c]{|>{\columncolor{lightgray}\color{black}}l|p{0.3\textwidth}|}
			\hline
			\textbf{Methods} & \textbf{Configurations} \\
			\hline
			\cellcolor{lightgray}AdamW & $learning\_rate=0.001$     \\
									&	$weight\_decay=0.01$ \\
			\hline
			\cellcolor{lightgray}ReduceLROnPlateau & $mode= 'max'$  \\
												&   $factor=0.1$   \\
												&  $patience=30$ \\
    			\hline
		\end{tabular}
	}
	\label{table:config1}
   \vspace{0.1\baselineskip}
\end{table}
	
\subsection{利用沒有標記的資料產生偽標記去訓練}
	此步驟主要想利用沒有被標記的資料提升模型的表現度,透過前一步驟訓練好的模型預測沒有標記的資料,當資料通過 softmax 的幾率或信心程度超過一個臨界值,我們就給他一個偽標記,並把他加入訓練資料裡進行訓練,不斷地重複預測、拿到偽標籤、訓練等步驟。跟前一步驟一樣,我採取了 AdamW 作為我的 optimizer, cross-entropy 作為我的損失函數,我同樣也採用了 ReduceLROnPlateau 的學習速率調整策略,根據我驗證集的表現進而調整學習速率。 Table \ref{table:config2} 同樣呈現了我在此階段訓練時 optimizer 以及 learning rate scheduler 的參數。
	
	對於臨界值的部分,我採取的是動態調整臨界值的方式,主要想法為當當前 epoch 驗證集的準確度超過過去之前所有 epoch 的驗證集最高準確度的話,我們就將臨界值調低,好讓更多未被標記的圖片可以成功被標記進而加進訓練資料訓練,反之則調高。 臨界值 $\gamma$ 的調整公式如下:

\begin{equation}
  \label{eqn:dt}  
  \gamma_{t+1} = \left\{\begin{array}{ll}
                 max(0.6, \gamma_t \times 0.95 ), & \mbox{if $valid\_acc > best\_acc$} \\ 
                 min(0.8, \gamma_t \times 1.05 ), & \mbox{otherwise,}  
                \end{array} \right. 
\end{equation}
	
	其中 $t$ 為第 $t$ 個 epoch, $valid\_acc$ 表示當前 epoch 驗證集的準確度, $best\_acc$ 則代表過去之前所有 epoch 的驗證集最高準確度。


\begin{table}[htb]
	\centering	
	\normalsize
    \newcommand{\z}{\phantom{0}}
    \caption{Configurations of optimizer and learning rate scheduler}
    \vspace{0.15\baselineskip}
    \resizebox{0.7\textwidth}{!}{
		\begin{tabular}[c]{|>{\columncolor{lightgray}\color{black}}l|p{0.3\textwidth}|}
			\hline
			\textbf{Methods} & \textbf{Configurations} \\
			\hline
			\cellcolor{lightgray}AdamW & $learning\_rate=0.001$     \\
									&	$weight\_decay=0.01$ \\
			\hline
			\cellcolor{lightgray}ReduceLROnPlateau & $mode= 'max'$  \\
												&   $factor=0.1$   \\
												&   $patience=7$ \\
												&    $cooldown=5$   \\ 
												&    $min_lr=10^{-15}$ \\
    			\hline
		\end{tabular}
	}
	\label{table:config2}
   \vspace{0.1\baselineskip}
\end{table}



\section{結論與發現}

以下是我針對本次實驗的心得整理以及發現:
\begin{enumerate}[label=\textbf{\arabic*.}]
\item   RMSProp 相較 Adam 收斂速度較慢。
\item	剛開始使用 Adam 時學習率調太高 (0.01 到 0.1 之間) 導致訓練結果較不佳。
\item	在參考老師的講義後,發覺 Adam 的學習率不能調太高。
\item	在試驗幾次後,決定以 AdamW 有考量 L2 正則化的情況下搭配 ReduceLROnPlateau 的學習率調整策略作為我最終的訓練方法可以達到較好的結果。
\item   使用動態調整臨界值時,相比只有訓練有標記的資料提升了 2\% - 3\% 的準確率。

\end{enumerate}


此外,我有實作不同 CNN 的模型,分別為 ResNet\ \cite{he2015deep} 以及 Inception-ResNet-v2\ \cite{szegedy2016inceptionv4},這裡註記在比較時沒有使用他們 pretrain 過後的模型或是權重去做比較,這裡發現使用大模型的架構時,可能反而表現較差,可能因為此次作業的資料為較小的資料集,因此可能在深的模型上表現較差。 Table\ \ref{table:comparison} 展示我的模型與其他模型的實驗數據比較。

\begin{table}[htb]
	\centering	
	\normalsize
    \newcommand{\z}{\phantom{0}}
    \caption{Simulation results of each model}
    \vspace{0.15\baselineskip}
    %\resizebox{0.5\textwidth}{!}{
	\begin{tabularx}{0.6\textwidth}{@{}lr@{}}\toprule
		\textbf{Model} & \textbf{Accuracy} \\
		\hline
		ResNet18 & 0.7582  \\ 
		Inception-ResNet-v2  & 0.7910  \\
		MyModel & 0.7817  \\
    		\hline

	\end{tabularx}
	%}
	\label{table:comparison}
   \vspace{0.15\baselineskip}
\end{table}








\bibliographystyle{IEEEtran} 
\bibliography{reference}

\end{document}